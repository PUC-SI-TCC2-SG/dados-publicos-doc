                                                                                       %%%%%%%%%%%%%%%%%%%%%%%%%%%%%%%%%%%%%%%%%%%%%%%%%%%%%%%%%%%%%%%%%%%%%%%%%%%%%%%%%%%%%%%%%%%%%%%%%%%%%%%
%%%%%%%%%%%%%% Template de Artigo Adaptado para Trabalho de Diplomação do ICEI %%%%%%%%%%%%%%%%%%%%%%%%
%% codificação UTF-8 - Abntex - Latex -  							     %%
%% Autor:    Paulo Cezar Paim do Carmo  (paulo.carmo@pucminas.br)                            %% 
%% Co-autor: Prof.                                      %%
%% Revisores normas NBR (Padrão PUC Minas):                   %%
%% Versão: 1.0                                                                       %%
%%%%%%%%%%%%%%%%%%%%%%%%%%%%%%%%%%%%%%%%%%%%%%%%%%%%%%%%%%%%%%%%%%%%%%%%%%%%%%%%%%%%%%%%%%%%%%%%%%%%%%%
\section{\esp Introdução}

A "Lei de Transparência"~\cite{LeiC131:2009} tem como objetivo de divulgar os gastos da União, dos Estados, Municípios e do Distrito Federal, e o decreto de  nº 8777~\cite{Decreto8777:2016} que tem o intuito de regulamentar a "Lei de Transparência", como promover a publicação de dados de órgãos e  entidades administrativas pública e federais, de modo que sejam publicados de modo contínuos sob a forma de dados abertos. O intuito de disponibilizar esses dado, é de tornar mais transparente dados públicos e ter a participação dos cidadãos, e poder gerar diversas aplicações desenvolvida pela sociedade.

No Intuito de cumprir essa lei o governo, criou o portal de Dados Abertos {Citar o site http://dados.gov.br/} para que os cidadãos possam acessar os dados para poder verificar os públicos criados e disponibilizado como arquivos .CSV e PDF. O senado federal também disponibilizou em seu portal{Cite o site https://www12.senado.leg.br/hpsenado}, dados para que possam acessar e visualizar os dados do senado. Para acessar esses dados o portal disponibiliza os dados administrativos e legislativo do senado que neles contém os dados financeiros quanto os dados de votação e comissão dos senadores. 

Atualmente existem muitos projetos que utilizam os dados públicos como o projeto serenata de amor{Cite o site https://serenata.ai/} que pega os dados do da câmara dos deputados e verifica através de uma inteligência artificial esses dados se tem alguma irregularidade nos lançamentos da câmara e pública em suas redes sociais para que possam verificar esses lançamentos. 

E quando você tenta acessar os dados de gastos do senadores, essas informações estão disponíveis em formato de texto que você tem que ver de um senador específico e não tem como ver de todos os senadores ao mesmo tempo para ter uma comparação, ou está disponível através de um arquivo csv com todos os dados dos senadores para que possa ver todos os dados, com isso dificultando a visualização dessas informações a pessoas que não tem muito conhecimento de informática.  

O  objetivo deste trabalho é desenvolver um sistema que coleta essas informações de gastos dos senadores e através desses dados possa gerar gráficos que resumem essas informações e com isso cidadãos que não tem muito conhecimento em informática possa visualizar esses dados públicos de forma mais fácil e intuitiva através de tabelas dinâmicas.




\section{\esp REFERENCIAL TEÓRICO}

Artigo Visualização de dados abertos no setor público ~\cite{Silva:2018} 
Neste artigo explica que o governo disponibiliza os dados públicos através de arquivos no formato .CSV e PDF, que muita das vezes, não fica claro para a população os dados presentes nesses arquivos. Com a ajuda de infográficos, é possível que esses dados sejam mais bem visualizado, e que tem ferramentas no mercado que podem facilitar essa visualização. O Site visme.co que foi utilizado nesse artigo, que pegou os dados de "Violências  Domésticas,  Sexual  e  Outras", e apresentou de forma mais fácil compreensão utilizando a prática chamada de narração de dados(Storytelling), que foi possível passar uma história através desses dados. 

\subsection{\esp Visualização de dados}



\subsection{\esp Ferramentas para visualização}



\subsection{\esp Visualização de dados públicos}
 
 
 

 
 

\section{\esp Metodologia}

\section{\esp Desenvolvimento}

\section{\esp Resultados}

\section{\esp Conclusão}









% \subsection{\esp Trabalhos futuros}
% 
% Sugestões de estudos posteriores são ser adicionados subseção deste capítulo de conclusão.